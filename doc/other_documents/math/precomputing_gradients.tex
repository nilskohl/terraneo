\documentclass{article}
\usepackage{amsmath}
\usepackage{parskip}
\usepackage{cleveref}

\title{Precomputing Gradients for the Spherical Shell}
\author{Nils Kohl}
\date{\today}

\begin{document}

\maketitle

\section{Introduction}

Let's consider the bilinear form for the Laplacian
\begin{equation}
    a(u, v) = \int_{\Omega} \nabla u \cdot \nabla v \, d\Omega,
\end{equation}
where $\Omega$ is the spherical shell.

On each element, we need to approximate integrals of the form
\begin{equation}
    \int_{T} \nabla \phi_i \cdot \nabla \phi_j \, dT,
\end{equation}
where $\phi_i$ and $\phi_j$ are the shape functions associated with the $i$-th and $j$-th nodes of the element $T$.

After transformation to the reference element, we can express this as
\begin{equation}
    \int_{T} \nabla \phi_i \cdot \nabla \phi_j = \int_{\hat{T}} J^{-T} \nabla N_i \cdot J^{-T} \nabla N_j |\det(J)| \, d\hat{T},
\end{equation} 
where $J$ is the Jacobian of the transformation from the reference element $\hat{T}$ to the physical element $T$, and $N_i$ and $N_j$ are the shape functions in the reference element.

We approximate the integral by quadrature. Consider some quadrature point $q_k$ with weight $w_k$. The integral becomes
\begin{equation}
    \int_{T} \nabla \phi_i \cdot \nabla \phi_j \approx \sum_k w_k J^{-T}(q_k) \nabla N_i(q_k) \cdot J^{-T}(q_k) \nabla N_j(q_k) |\det(J(q_k))|.
\end{equation}

In a matrix-free setting, it is beneficial to exploit structure to avoid recomputing the gradients of the shape functions at each quadrature point for each element. 

\section{Splitting the Jacobian}

We know that the physical elements are extruded in radial direction from 2D elements. It comes to mind to somehow split the Jacobian into a radial part and a 2D part, such that we can precompute the gradients of the shape functions in the 2D reference element.

I manage to arrive at a splitting of the Jacobian that looks like this
\begin{equation}
    J = J_\mathrm{lat} D_\mathrm{rad}
\end{equation}
where $J_\mathrm{lat}$ does not depend on the radial component of the physical element coordinates, and $D_\mathrm{rad}$ is a diagonal matrix.
Specifically, we use the forward map 
\begin{equation}
    \label{eq:forward_map}
    x(\xi, \eta, \zeta) = r(\zeta) X_\mathrm{surf}(\xi, \eta),
\end{equation}
where $X_\mathrm{surf} = [x_s(\xi,\eta), y_s(\xi,\eta), z_s(\xi,\eta)]^T$.
Importantly, $X_\mathrm{surf}$ is on the surface of the spherical shell (or the unit sphere), i.e., it does not depend on the radial coordinate $\zeta$ and also not on the "physical radial coordinate".

For the Jacobian, $J$ we compute
\begin{align}
    % ∂x/∂ξ = ∂/∂ξ [ r(ζ) * X_surf(ξ, η) ]
    \frac{\partial x}{\partial \xi} &= \frac{\partial}{\partial \xi} (r(\zeta) X_\mathrm{surf}(\xi, \eta)) 
    = r(\zeta) \frac{\partial}{\partial \xi} X_\mathrm{surf}(\xi, \eta) \\
    \frac{\partial x}{\partial \eta} &= \frac{\partial}{\partial \eta} (r(\zeta) X_\mathrm{surf}(\xi, \eta)) 
    = r(\zeta) \frac{\partial}{\partial \eta} X_\mathrm{surf}(\xi, \eta) \\
    \frac{\partial x}{\partial \zeta} &= \frac{\partial}{\partial \zeta} (r(\zeta) X_\mathrm{surf}(\xi, \eta)) 
    =  (\frac{\partial}{\partial \zeta} r(\zeta)) X_\mathrm{surf}(\xi, \eta)
\end{align}
and thus
\begin{equation}
    J = \begin{pmatrix}
        r(\zeta) \frac{\partial X_\mathrm{surf}}{\partial \xi} & r(\zeta) \frac{\partial X_\mathrm{surf}}{\partial \eta} & \frac{\partial r}{\partial \zeta} X_\mathrm{surf}
    \end{pmatrix}.
\end{equation}
This can be split now into a 2D part and a radial part:
\begin{equation}
    J = \begin{pmatrix}
        \frac{\partial X_\mathrm{surf}}{\partial \xi} & \frac{\partial X_\mathrm{surf}}{\partial \eta} &  X_\mathrm{surf}
    \end{pmatrix} \begin{pmatrix}
        r(\zeta) & 0 & 0 \\
        0 & r(\zeta) & 0 \\
        0 & 0 & \frac{\partial r}{\partial \zeta}
    \end{pmatrix} = J_\mathrm{lat} D_\mathrm{rad}.
\end{equation}

It would be nice if we could precompute a term like $J_\mathrm{lat}^{-T}(q_k) \nabla N_i(q_k)$ and then scale with $D_\mathrm{rad}$.

Unfortunately, this is not straightforward since we arrive at
\begin{equation}
    J^{-T} \nabla N_i = J_\mathrm{lat}^{-T} D_\mathrm{rad}^{-1} \nabla N_i.
\end{equation}
and we cannot precompute $D_\mathrm{rad}^{-1}$ since it depends on the radial coordinate of the nodes of the physical element. (In general, matrix-matrix multiplication with diagonal matrices does not commute.)

What we can (and will) do instead: we split the gradient up further.

\section{(Almost) Precomputing the Physical Gradients}

We consider shape functions that are defined through a tensor product of a radial part $N_i^\mathrm{rad}$ and a 2D part $N_i^\mathrm{lat}$, i.e., we have
\begin{equation}
    \label{eq:shape_function}
    N_i(\xi, \eta, \zeta) = N_i^\mathrm{lat}(\xi, \eta) \, N_i^\mathrm{rad}(\zeta),
\end{equation}
with gradient
\begin{equation}
    \nabla N_i = \left( \frac{\partial N_i}{\partial \xi}, \frac{\partial N_i}{\partial \eta}, \frac{\partial N_i}{\partial \zeta} \right)^T.
\end{equation}
The components of the gradient are
\begin{align}
    \frac{\partial N_i}{\partial \xi} &= \frac{\partial N_i^\mathrm{lat}}{\partial \xi} N_i^\mathrm{rad}, \\
    \frac{\partial N_i}{\partial \eta} &= \frac{\partial N_i^\mathrm{lat}}{\partial \eta} N_i^\mathrm{rad}, \\
    \frac{\partial N_i}{\partial \zeta} &= N_i^\mathrm{lat} \frac{\partial N_i^\mathrm{rad}}{\partial \zeta}.
\end{align}

This provides the isolation we need to precompute the gradients of the shape functions in the 2D reference element.

We want to compute
\begin{equation}
    J^{-T} \nabla N_i = J_\mathrm{lat}^{-T} D_\mathrm{rad}^{-1} \nabla N_i.
\end{equation}

Let's start with 
\begin{equation}
    D_\mathrm{rad}^{-1} \nabla N_i = \begin{bmatrix}
        \frac{1}{r(\zeta)} \frac{\partial N_i^\mathrm{lat}}{\partial \xi} N_i^\mathrm{rad} \\
        \frac{1}{r(\zeta)} \frac{\partial N_i^\mathrm{lat}}{\partial \eta} N_i^\mathrm{rad} \\
        \frac{1}{\frac{\partial r}{\partial \zeta}} N_i^\mathrm{lat} \frac{\partial N_i^\mathrm{rad}}{\partial \zeta}
    \end{bmatrix}
    =
    \frac{1}{r(\zeta)}
    \begin{bmatrix}
        \frac{\partial N_i^\mathrm{lat}}{\partial \xi} N_i^\mathrm{rad} \\
        \frac{\partial N_i^\mathrm{lat}}{\partial \eta} N_i^\mathrm{rad} \\
        0
    \end{bmatrix}
    +
    \frac{\partial N_i^\mathrm{rad}}{\partial \zeta} \frac{1}{\frac{\partial r}{\partial \zeta}}
    \begin{bmatrix}
        0 \\
        0 \\
        N_i^\mathrm{lat}
    \end{bmatrix}.
\end{equation}

Then we multiply with $J_\mathrm{lat}^{-T}$ from the left:
\begin{equation}
    J_\mathrm{lat}^{-T} D_\mathrm{rad}^{-1} \nabla N_i =
    \frac{1}{r(\zeta)}  
    \underbrace{
    J_\mathrm{lat}^{-T}
    \begin{bmatrix}  
        \frac{\partial N_i^\mathrm{lat}}{\partial \xi} N_i^\mathrm{rad} \\
        \frac{\partial N_i^\mathrm{lat}}{\partial \eta} N_i^\mathrm{rad} \\
        0
    \end{bmatrix}
    }_{=:\, G_i^\mathrm{rad}(\xi, \eta)}
    +
    \frac{\partial N_i^\mathrm{rad}}{\partial \zeta} \frac{1}{\frac{\partial r}{\partial \zeta}} 
    \underbrace{
    J_\mathrm{lat}^{-T}
    \begin{bmatrix}
        0 \\
        0 \\
        N_i^\mathrm{lat}
    \end{bmatrix}
    }_{=:\, G_i^\mathrm{lat}(\xi, \eta)}.
\end{equation}

Putting everything together, we have
\begin{align}
    \int_{T} \nabla \phi_i \cdot \nabla \phi_j &\\ 
    &\approx \sum_k w_k J^{-T}(q_k) \nabla N_i(q_k) \cdot J^{-T}(q_k) \nabla N_j(q_k) |\det(J(q_k))| \\
    &= \sum_k w_k \left(\frac{1}{r(q_{k,\zeta})} G_i^{\mathrm{rad}}(q_{k,\xi}, q_{k,\eta}) + 
    \frac{\partial N_i^\mathrm{rad}}{\partial \zeta} \frac{1}{\frac{\partial r}{\partial \zeta}} G_i^{\mathrm{lat}}(q_{k,\xi}, q_{k,\eta})\right) \cdot \\
    & \left(\frac{1}{r(q_{k,\zeta})} G_j^{\mathrm{rad}}(q_{k,\xi}, q_{k,\eta}) + 
    \frac{\partial N_j^\mathrm{rad}}{\partial \zeta} \frac{1}{\frac{\partial r}{\partial \zeta}} G_j^{\mathrm{lat}}(q_{k,\xi}, q_{k,\eta})\right) |\det(J(q_k))| 
\end{align}
where $q_k = (q_{k,\xi}, q_{k,\eta}, q_{k,\zeta})^T$.

Finally must not forget the determinant of the Jacobian, which can be split as well:
\begin{equation}
    |\det(J(q_k))| = r(q_{k,\zeta})^2 \frac{\partial r}{\partial \zeta} \underbrace{|\det(J_\mathrm{lat}(q_{k,\xi}, q_{k,\eta}))|}_{G^\mathrm{det}(q_{k,\xi}, q_{k,\eta})}.
\end{equation}
($\frac{\partial r}{\partial \zeta} > 0$ by construction, see \cref{sec:concrete_implementation}.)

We can precompute the terms $G_i^{\mathrm{rad}}(q_{k,\xi}, q_{k,\eta})$ and $G_i^{\mathrm{lat}}(q_{k,\xi}, q_{k,\eta})$ for all shape functions $i$ and quadrature points $k$. For tensor product quadrature, this reduces the number of quadrature points we need to precompute this for. Different approaches are possible: either we precompute and store
these vectors a priori, or we compute them on-the-fly but cache them for all elements in radial direction (i.e., we loop over the radially aligned elements in the "innermost" loop).

Obviously can be applied to any bilinear form that includes the gradient of the shape functions, not just the Laplacian.

\section{Concrete Implementation}
\label{sec:concrete_implementation}

The above framework should be applicable to both hexahedral and prismatic (wedge) elements. Let's consider the case of prismatic elements, which are defined by a triangular 2D surface element and a radial extrusion. (Note: the triangular
surface element is not curved in this case, but I suppose it could be curved as well.)

The shape functions on the reference wedge
\begin{equation}
     [(0, 0, -1), (1, 0, -1), (0, 1, -1), (0, 0, 1), (1, 0, 1), (0, 1, 1)]
\end{equation}
are given by \cref{eq:shape_function} with
\begin{align}
    N_1^\mathrm{lat}(\xi, \eta) = N_4^\mathrm{lat}(\xi, \eta) &= 1 - \xi - \eta, \\
    N_2^\mathrm{lat}(\xi, \eta) = N_5^\mathrm{lat}(\xi, \eta) &= \xi, \\
    N_3^\mathrm{lat}(\xi, \eta) = N_6^\mathrm{lat}(\xi, \eta) &= \eta,
\end{align}
and
\begin{align}
    N_1^\mathrm{rad}(\zeta) = N_2^\mathrm{rad}(\zeta) = N_3^\mathrm{rad}(\zeta) &= \frac{1}{2}(1 - \zeta) \\
    N_4^\mathrm{rad}(\zeta) = N_5^\mathrm{rad}(\zeta) = N_6^\mathrm{rad}(\zeta) &= \frac{1}{2}(1 + \zeta).
\end{align}
The gradients of the shape functions are then given by
\begin{align}
    \frac{\partial N^\mathrm{lat}_1}{\partial \xi} &= -1, & \frac{\partial N^\mathrm{lat}_1}{\partial \eta} &= -1, & \frac{\partial N^\mathrm{rad}_1}{\partial \zeta} &= -\frac{1}{2}, \\
    \frac{\partial N^\mathrm{lat}_2}{\partial \xi} &= 1, & \frac{\partial N^\mathrm{lat}_2}{\partial \eta} &= 0, & \frac{\partial N^\mathrm{rad}_2}{\partial \zeta} &= -\frac{1}{2}, \\
    \frac{\partial N^\mathrm{lat}_3}{\partial \xi} &= 0, & \frac{\partial N^\mathrm{lat}_3}{\partial \eta} &= 1, & \frac{\partial N^\mathrm{rad}_3}{\partial \zeta} &= -\frac{1}{2}, \\
    \frac{\partial N^\mathrm{lat}_4}{\partial \xi} &= -1, & \frac{\partial N^\mathrm{lat}_4}{\partial \eta} &= -1, & \frac{\partial N^\mathrm{rad}_4}{\partial \zeta} &= \frac{1}{2}, \\    
    \frac{\partial N^\mathrm{lat}_5}{\partial \xi} &= 1, & \frac{\partial N^\mathrm{lat}_5}{\partial \eta} &= 0, & \frac{\partial N^\mathrm{rad}_5}{\partial \zeta} &= \frac{1}{2}, \\
    \frac{\partial N^\mathrm{lat}_6}{\partial \xi} &= 0, & \frac{\partial N^\mathrm{lat}_6}{\partial \eta} &= 1, & \frac{\partial N^\mathrm{rad}_6}{\partial \zeta} &= \frac{1}{2}.
\end{align}

It remains to compute the Jacobian. For that we need the forward map (\cref{eq:forward_map}) of the reference wedge to the physical wedge. Let the physical wedge $[p_1, \ldots, p_6]$ be defined by the three points $p^s_k = (x_k, y_k, z_k)^T, k = 1, 2, 3$ on the unit sphere and two radii $r_1, r_2, r_1 < r_2$:
\begin{align}
    p_k := r_1 p^s_k, & \quad k = 1, 2, 3, \\
    p_{k+3} := r_2 p^s_k, & \quad k = 1, 2, 3.
\end{align}

We get the forward map \cref{eq:forward_map} with
\begin{equation}
    r(\zeta) = r_1 + \frac{r_2 - r_1}{2} (1 + \zeta),
\end{equation}
and the surface points
\begin{align}
    X_\mathrm{surf}(\xi, \eta) &= (1 - \xi - \eta) p^s_1 + \xi p^s_2 + \eta p^s_3, \\
    &= (1 - \xi - \eta) (x_1, y_1, z_1)^T + \xi (x_2, y_2, z_2)^T + \eta (x_3, y_3, z_3)^T.
\end{align}

To complete the Jacobian, we compute the partial derivatives:
\begin{align}
    \frac{\partial X_\mathrm{surf}}{\partial \xi} &= - (x_1, y_1, z_1)^T + (x_2, y_2, z_2)^T, \\
    \frac{\partial X_\mathrm{surf}}{\partial \eta} &= - (x_1, y_1, z_1)^T + (x_3, y_3, z_3)^T, \\
    \frac{\partial r}{\partial \zeta} &= \frac{r_2 - r_1}{2}.
\end{align}



\end{document}